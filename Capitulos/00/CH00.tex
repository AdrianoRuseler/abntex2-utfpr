

\cleardoublepage
\phantomsection %The \phantomsection command is needed to create a link to a place in the document that is not a figure, equation, table, section, subsection, chapter, etc.
\addcontentsline{toc}{chapter}{\texorpdfstring{\MakeTextUppercase{Capítulo 1}}{Capítulo 1}} 



\chapter[Introdução]{Introdução \showfont}
%\addcontentsline{toc}{chapter}{Introdução}
%\texorpdfstring{\MakeTextUppercase{Capítulo 1}}{Capítulo 1}
% ----------------------------------------------------------




	% Write epigraphs
\begin{flushright}
	\textit{``The curious paradox is that when I accept myself just as I am, then I can change.''}\\
	Carl Rogers
\end{flushright}



\showfont
\newpage


\section[Some encoding tests]{\showfont}
\subsection{\showfont}
\subsubsection{\showfont}
\subsubsubsection{\showfont}


\textsf{textsf: \showfont} 

\textrm{textrm: \showfont}

\textnormal{textnormal: \showfont}

\textbf{textbf: \showfont} 

\textit{textit: \showfont}

footnote\footnote{\showfont}

% \emph{emphasize: \showfont} 

``Modelo Canônico \showfont''

\begin{itemize}
	\item \textrm{Roman family - \showfont }
	\item \textsf{Sans serif family - \showfont}
	\item \texttt{Typewriter/teletype family - \showfont}
	\item \textit{italics text - \showfont}
	\item \textsl{slanted text- \showfont}
	\item \textsc{small caps text- \showfont}	
\end{itemize}


\begin{equation}
\showfont
\end{equation}

mathnormal -  default: $\mathnormal{abcXYZ \justshowfont }$

mathrm - roman: $\mathrm{abcXYZ \justshowfont }$

mathbf - bold roman: $\mathbf{abcXYZ \justshowfont }$

mathsf - sans serif: $\mathsf{abcXYZ \justshowfont }$

mathit - text italic: $\mathit{abcXYZ \justshowfont }$

mathtt -  typewriter: $\mathtt{abcXYZ \justshowfont }$

mathcal - calligraphic: $\mathcal{XYZ}$


%\pagevalues

\begin{figure}
	\caption{Page layout for this document -  \showfont} \label{fig:ptrs}
	%\drawpage
	\setlayoutscale{0.4}
	%	\currentpage
	\drawparameterstrue
	\drawpage
	%	\pagedesign
	%	\pagevalues
\end{figure}


\begin{figure}
	\caption{Page layout values for this document} \label{fig:ptrsval}
	
	\pagevalues
\end{figure}



\begin{figure}
	\currentfootnote
	\drawparameterstrue
	%	\setlayoutscale{0.4}
	\drawfootnote
	\footnotevalues
	\caption{The current footnote layout}\label{fig:ftry}
\end{figure}	
	

\begin{figure}
	\drawparagraph
	\paragraphvalues
	\caption{Paragraph parameters}\label{fig:fpara}
\end{figure}


\begin{figure}
	\setlayoutscale{0.6}
	\drawparameterstrue
	\drawtoc
	\tocvalues
	\caption{Table of Contents entry parameters}\label{fig:tocp}
\end{figure}




Este documento e seu código-fonte são exemplos de referência de uso da classe
\textsf{abntex2} e do pacote \textsf{abntex2cite}. O documento 
exemplifica a elaboração de trabalho acadêmico (tese, dissertação e outros do
gênero) produzido conforme a ABNT NBR 14724:2011 \emph{Informação e documentação
- Trabalhos acadêmicos - Apresentação}.

A expressão ``Modelo Canônico'' é utilizada para indicar que \abnTeX\ não é
modelo específico de nenhuma universidade ou instituição, mas que implementa tão
somente os requisitos das normas da ABNT. Uma lista completa das normas
observadas pelo \abnTeX\ é apresentada em \citeonline{abntex2classe}.

Sinta-se convidado a participar do projeto \abnTeX! Acesse o site do projeto em
\url{http://abntex2.googlecode.com/}. Também fique livre para conhecer,
estudar, alterar e redistribuir o trabalho do \abnTeX, desde que os arquivos
modificados tenham seus nomes alterados e que os créditos sejam dados aos
autores originais, nos termos da ``The \LaTeX\ Project Public
License''\footnote{\url{http://www.latex-project.org/lppl.txt}}.

Encorajamos que sejam realizadas customizações específicas deste exemplo para
universidades e outras instituições --- como capas, folha de aprovação, etc.
Porém, recomendamos que ao invés de se alterar diretamente os arquivos do
\abnTeX, distribua-se arquivos com as respectivas customizações.
Isso permite que futuras versões do \abnTeX~não se tornem automaticamente
incompatíveis com as customizações promovidas. Consulte
\citeonline{abntex2-wiki-como-customizar} par mais informações.

Este documento deve ser utilizado como complemento dos manuais do \abnTeX\ 
\cite{abntex2classe,abntex2cite,abntex2cite-alf} e da classe \textsf{memoir}
\cite{memoir}. 

Esperamos, sinceramente, que o \abnTeX\ aprimore a qualidade do trabalho que
você produzirá, de modo que o principal esforço seja concentrado no principal:
na contribuição científica.

Equipe \abnTeX 

Lauro César Araujo


