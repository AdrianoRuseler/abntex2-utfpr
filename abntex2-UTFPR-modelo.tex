%% abtex2-modelo-trabalho-academico.tex, v-1.9.6 laurocesar
%% Copyright 2012-2016 by abnTeX2 group at http://www.abntex.net.br/ 
%%
%% This work may be distributed and/or modified under the
%% conditions of the LaTeX Project Public License, either version 1.3
%% of this license or (at your option) any later version.
%% The latest version of this license is in
%%   http://www.latex-project.org/lppl.txt
%% and version 1.3 or later is part of all distributions of LaTeX
%% version 2005/12/01 or later.
%%
%% This work has the LPPL maintenance status `maintained'.
%% 
%% The Current Maintainer of this work is the abnTeX2 team, led
%% by Lauro César Araujo. Further information are available on 
%% http://www.abntex.net.br/
%%
%% This work consists of the files abntex2-modelo-trabalho-academico.tex,
%% abntex2-modelo-include-comandos and abntex2-modelo-references.bib
%%

% ------------------------------------------------------------------------
% ------------------------------------------------------------------------
% abnTeX2: Modelo de Trabalho Academico (tese de doutorado, dissertacao de
% mestrado e trabalhos monograficos em geral) em conformidade com 
% ABNT NBR 14724:2011: Informacao e documentacao - Trabalhos academicos -
% Apresentacao
% ------------------------------------------------------------------------
% ------------------------------------------------------------------------

\documentclass[
	% -- opções da classe memoir --
	12pt,				% tamanho da fonte
	openright,			% capítulos começam em pág ímpar (insere página vazia caso preciso)
	twoside,			% para impressão em recto e verso. Oposto a oneside
	a4paper,			% tamanho do papel. 
	% -- opções da classe abntex2 --
	%chapter=TITLE,		% títulos de capítulos convertidos em letras maiúsculas
	%section=TITLE,		% títulos de seções convertidos em letras maiúsculas
	%subsection=TITLE,	% títulos de subseções convertidos em letras maiúsculas
	%subsubsection=TITLE,% títulos de subsubseções convertidos em letras maiúsculas
	%sumario=tradicional,
	sumario=abnt-6027-2012, % memoir v3.6k ou superior
	% -- opções do pacote babel --
	english,			% idioma adicional para hifenização
	brazil				% o último idioma é o principal do documento
	]{abntex2-utfpr}


% ---
% Pacotes de citações
% ---
\usepackage[brazilian,hyperpageref]{backref}	 % Paginas com as citações na bibl
\usepackage[alf]{abntex2cite}	% Citações padrão ABNT
%\usepackage[num]{abntex2cite} % Citação numérica [num]
%\citebrackets[]


% Altera de sin e tan para sen e tg nas equações
\DeclareMathOperator{\sen}{sen}
\renewcommand{\sin}{\sen} 

\DeclareMathOperator{\tg}{tg}
\renewcommand{\tan}{\tg} 

% % Para uso futuro
%\usepackage[
%language = brazil,
%style = abnt, % Sistema alfabético
%% style = abnt-numeric, % Sistema numérico
%% style = abnt-ibid, % Notas de referência
%]{biblatex}
%


% --- 
% CONFIGURAÇÕES DE PACOTES
% --- 

% ---
% Configurações do pacote backref
% Usado sem a opção hyperpageref de backref
\renewcommand{\backrefpagesname}{Citado na(s) página(s):~}
% Texto padrão antes do número das páginas
\renewcommand{\backref}{}
% Define os textos da citação
\renewcommand*{\backrefalt}[4]{
	\ifcase #1 %
		Nenhuma citação no texto.%
	\or
		Citado na página #2.%
	\else
		Citado #1 vezes nas páginas #2.%
	\fi}%
% ---

% ---
% Informações de dados para CAPA e FOLHA DE ROSTO
% ---
\titulo{Modelo Canônico de\\ Trabalho Acadêmico com \abnTeX}
\autor{Nome do Autor}

%\autorA{Nome do Autor A} % Em caso de TCC
%\autorB{Nome do Autor B} % Em caso de TCC
%\autorC{Nome do Autor C} % Em caso de TCC

\local{Curitiba}
\data{\today}
\orientador{Nome do orientador}
\coorientador{Equipe \abnTeX}

\instituicao{Universidade Tecnológica Federal do Paraná}
\departamento{Departamento Acadêmico de Eletrotécnica}
\programa{Programa de Pós--Graduação em Engenharia Elétrica e Informática Industrial}


\siglainstituicao{UTFPR}
\sigladepartamento{DAELT}
\siglaprograma{CPGEI}


\tipotrabalho{Tipo do Trabalho} % TCC, Dissertação ou Tese
% O preambulo deve conter o tipo do trabalho, o objetivo, 
% o nome da instituição e a área de concentração 
\preambulo{Modelo canônico de trabalho monográfico acadêmico em conformidade com
as normas ABNT apresentado à comunidade de usuários \LaTeX.}
% ---



% ---
% Configurações de aparência do PDF final
% ---

%\definecolor{figcolor}{rgb}{1,0.4,0}  % orange
%\definecolor{tabcolor}{rgb}{1,0.4,0}  % orange
%\definecolor{eqncolor}{rgb}{1,0.4,0}  % orange
\definecolor{linkcolor}{rgb}{1,0.4,0}  % orange
%\definecolor{citecolor}{rgb}{1,0.4,0}  % orange
%\definecolor{seccolor}{rgb}{0,0,1}  % blue
%\definecolor{abscolor}{rgb}{0,0,1}  % blue
%\definecolor{titlecolor}{rgb}{0,0,1}  % blue
%\definecolor{biocolor}{rgb}{0,0,1}  % blue

% alterando o aspecto da cor azul
\definecolor{blue}{RGB}{41,5,195}

% informações do PDF
\makeatletter
\hypersetup{
     	%pagebackref=true,
		pdftitle={\@title}, 
		pdfauthor={\@author},
    	pdfsubject={\imprimirpreambulo},
	    pdfcreator={LaTeX with abnTeX2-utfpr},
		pdfkeywords={latex}{abntex}{abntex2}{\imprimirtipotrabalho}, 
		colorlinks=true,       		% false: boxed links; true: colored links
    	linkcolor=blue,          	% color of internal links
    	citecolor=blue,        		% color of links to bibliography
    	filecolor=magenta,      		% color of file links
		urlcolor=blue,
		bookmarksdepth=4
}
\makeatother
% --- 



% ---
% Estilo de capítulos
%
%\chapterstyle{default}
%\chapterstyle{pedersen} 
%\chapterstyle{lyhne} 
\chapterstyle{madsen} 
% \chapterstyle{veelo} 
%\chapterstyle{companion}

%\chapterstyle{thatcher}
%\chapterstyle{verville}
%\chapterstyle{VZ14} % Ver classe para maiores detalhes


%
% Veja outros estilos em:
% http://mirror.utexas.edu/ctan/info/latex-samples/MemoirChapStyles/MemoirChapStyles.pdf
% ---


% --- 
% Espaçamentos entre linhas e parágrafos 
% --- 

% O tamanho do parágrafo é dado por:
\setlength{\parindent}{1.3cm}

% Controle do espaçamento entre um parágrafo e outro:
\setlength{\parskip}{0.0cm}  % tente também \onelineskip

% ---
% compila o indice
% ---
\makeindex
% ---



%\includeonly{PreTexto/fichacatalografica}
%\includeonly{PreTexto/agradecimentos}
%\includeonly{PreTexto/resumos}
%\includeonly{PreTexto/siglas}
%\includeonly{PreTexto/simbolos}
%
%\includeonly{Capitulos/00/CH00}
%\includeonly{Capitulos/01/CH01}
%\includeonly{Capitulos/02/CH02}
%\includeonly{Capitulos/03/CH03}
%\includeonly{Capitulos/04/CH04}



% ----
% Início do documento
% ----
\begin{document}

% Seleciona o idioma do documento (conforme pacotes do babel)
%\selectlanguage{english}
\selectlanguage{brazil}

% Retira espaço extra obsoleto entre as frases.
\frenchspacing 

% ----------------------------------------------------------
% ELEMENTOS PRÉ-TEXTUAIS
% ----------------------------------------------------------
% \pretextual


% ---
% Capa
% ---
\imprimircapaPOS
%\imprimircapaGRAD

%\imprimircapaPOSnoBack % Capa sem imagem de fundo
%\imprimircapaGRADnoBack % Capa sem imagem de fundo


%\imprimircapautfpr
% ---

% ---
% Folha de rosto
% (o * indica que haverá a ficha bibliográfica)
% ---
\imprimirfolhaderosto*


% ---

% ---
% Inserir a ficha bibliografica
% ---

% Isto é um exemplo de Ficha Catalográfica, ou ``Dados internacionais de
% catalogação-na-publicação''. Você pode utilizar este modelo como referência. 
% Porém, provavelmente a biblioteca da sua universidade lhe fornecerá um PDF
% com a ficha catalográfica definitiva após a defesa do trabalho. Quando estiver
% com o documento, salve-o como PDF no diretório do seu projeto e substitua todo
% o conteúdo de implementação deste arquivo pelo comando abaixo:
%
% \begin{fichacatalografica}
%     \includepdf{fig_ficha_catalografica.pdf}
% \end{fichacatalografica}

\include{PreTexto/fichacatalografica}

%\include{PreTexto/errata}

% ---
% Inserir folha de aprovação
% ---

% Isto é um exemplo de Folha de aprovação, elemento obrigatório da NBR
% 14724/2011 (seção 4.2.1.3). Você pode utilizar este modelo até a aprovação
% do trabalho. Após isso, substitua todo o conteúdo deste arquivo por uma
% imagem da página assinada pela banca com o comando abaixo:
%
% \includepdf{folhadeaprovacao_final.pdf}
%


% ---
% Dedicatória
% ---
\include{PreTexto/dedicatoria}
% ---

% ---
% Agradecimentos
% ---
\include{PreTexto/agradecimentos}
% ---

% ---
% Epígrafe
% ---
\include{PreTexto/epigrafe}
% ---

% ---
% RESUMOS
% ---
\include{PreTexto/resumos}

% ---
% inserir lista de ilustrações
% ---
\pdfbookmark[0]{\listfigurename}{lof}
\listoffigures*
\cleardoublepage
% ---

% ---
% inserir lista de tabelas
% ---
\pdfbookmark[0]{\listtablename}{lot}
\listoftables*
\cleardoublepage
% ---


% ---
% inserir códigos fonte
% ---
\pdfbookmark[0]{\lstlistingname}{lol}
\lstlistoflistings* 
\cleardoublepage
% ---


% ---
% inserir lista de abreviaturas e siglas
% ---
\include{PreTexto/siglas}
% ---

% ---
% inserir lista de símbolos
% ---
\include{PreTexto/simbolos}
% ---

% ---
% inserir o sumario
% ---
\pdfbookmark[0]{\contentsname}{toc}
\tableofcontents*
\cleardoublepage
% ---



% ----------------------------------------------------------
% ELEMENTOS TEXTUAIS
% ----------------------------------------------------------
\textual

% ----------------------------------------------------------
% Introdução (exemplo de capítulo sem numeração, mas presente no Sumário)
% ----------------------------------------------------------
\include{Capitulos/00/CH00}


% ----------------------------------------------------------
% PARTE
% ----------------------------------------------------------
%\part{Preparação da pesquisa}
% ----------------------------------------------------------

% ---
% Capitulo com exemplos de comandos inseridos de arquivo externo 
% ---
\include{Capitulos/01/CH01}
% ---

\include{Capitulos/02/CH02}
\include{Capitulos/03/CH03}
\include{Capitulos/04/CH04}

% ----------------------------------------------------------
% Finaliza a parte no bookmark do PDF
% para que se inicie o bookmark na raiz
% e adiciona espaço de parte no Sumário
% ----------------------------------------------------------
\phantompart

% ---
% Conclusão
% ---
\include{Capitulos/Concl/Concl}
% ---

% ----------------------------------------------------------
% ELEMENTOS PÓS-TEXTUAIS
% ----------------------------------------------------------
\postextual
% ----------------------------------------------------------

% ----------------------------------------------------------
% Referências bibliográficas
% ----------------------------------------------------------
\bibliography{abntex2-modelo-references}

% ----------------------------------------------------------
% Glossário
% ----------------------------------------------------------
%
% Consulte o manual da classe abntex2 para orientações sobre o glossário.
%
%\glossary

% ----------------------------------------------------------
% Apêndices
% ----------------------------------------------------------

% ---
% Inicia os apêndices
% ---
\begin{apendicesenv}

% Imprime uma página indicando o início dos apêndices
\partapendices

	\include{Apendices/A/APA}
%	\include{Apendices/B/APB}
%	\include{Apendices/C/APC}

\end{apendicesenv}
% ---


% ----------------------------------------------------------
% Anexos
% ----------------------------------------------------------

% ---
% Inicia os anexos
% ---
\begin{anexosenv}

% Imprime uma página indicando o início dos anexos
\partanexos

	\include{Anexos/A/ANA}
%	\include{Anexos/B/ANB}
%	\include{Anexos/C/ANC}

\end{anexosenv}

%---------------------------------------------------------------------
% INDICE REMISSIVO
%---------------------------------------------------------------------
\phantompart
\printindex
%---------------------------------------------------------------------

\end{document}
