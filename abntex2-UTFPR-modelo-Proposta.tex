%% abtex2-modelo-trabalho-academico.tex, v-1.9.6 laurocesar
%% Copyright 2012-2016 by abnTeX2 group at http://www.abntex.net.br/ 
%%
%% This work may be distributed and/or modified under the
%% conditions of the LaTeX Project Public License, either version 1.3
%% of this license or (at your option) any later version.
%% The latest version of this license is in
%%   http://www.latex-project.org/lppl.txt
%% and version 1.3 or later is part of all distributions of LaTeX
%% version 2005/12/01 or later.
%%
%% This work has the LPPL maintenance status `maintained'.
%% 
%% The Current Maintainer of this work is the abnTeX2 team, led
%% by Lauro César Araujo. Further information are available on 
%% http://www.abntex.net.br/
%%
%% This work consists of the files abntex2-modelo-trabalho-academico.tex,
%% abntex2-modelo-include-comandos and abntex2-modelo-references.bib
%%

% ------------------------------------------------------------------------
% ------------------------------------------------------------------------
% abnTeX2: Modelo de Trabalho Academico (tese de doutorado, dissertacao de
% mestrado e trabalhos monograficos em geral) em conformidade com 
% ABNT NBR 14724:2011: Informacao e documentacao - Trabalhos academicos -
% Apresentacao
% ------------------------------------------------------------------------
% ------------------------------------------------------------------------

\documentclass[
	% -- opções da classe memoir --
	12pt,				% tamanho da fonte
	openright,			% capítulos começam em pág ímpar (insere página vazia caso preciso)
	twoside,			% para impressão em recto e verso. Oposto a oneside
	a4paper,			% tamanho do papel. 
	% -- opções da classe abntex2 --
	%chapter=TITLE,		% títulos de capítulos convertidos em letras maiúsculas
	%section=TITLE,		% títulos de seções convertidos em letras maiúsculas
	%subsection=TITLE,	% títulos de subseções convertidos em letras maiúsculas
	%subsubsection=TITLE,% títulos de subsubseções convertidos em letras maiúsculas
	%sumario=tradicional,
%	sumario=abnt-6027-2012, % memoir v3.6k ou superior
	% -- opções do pacote babel --
	english,			% idioma adicional para hifenização
	brazil				% o último idioma é o principal do documento
	]{abntex2-utfpr}


% ---
% Pacotes de citações
% ---
\usepackage[brazilian,hyperpageref]{backref}	 % Paginas com as citações na bibl
\usepackage[alf]{abntex2cite}	% Citações padrão ABNT
%\usepackage[num]{abntex2cite} % Citação numérica [num]
%\citebrackets[]



% % Para uso futuro
%\usepackage[
%language = brazil,
%style = abnt, % Sistema alfabético
%% style = abnt-numeric, % Sistema numérico
%% style = abnt-ibid, % Notas de referência
%]{biblatex}
%


% --- 
% CONFIGURAÇÕES DE PACOTES
% --- 

% ---
% Configurações do pacote backref
% Usado sem a opção hyperpageref de backref
\renewcommand{\backrefpagesname}{Citado na(s) página(s):~}
% Texto padrão antes do número das páginas
\renewcommand{\backref}{}
% Define os textos da citação
\renewcommand*{\backrefalt}[4]{
	\ifcase #1 %
	Nenhuma citação no texto.%
	\or
	Citado na página #2.%
	\else
	Citado #1 vezes nas páginas #2.%
	\fi}%
% ---



% ---
% Informações de dados para CAPA e FOLHA DE ROSTO
% ---
\titulo{Modelo Canônico de Trabalho Acadêmico com \abnTeX}
\tituloEng{Canonic Model of Academic Work with \abnTeX}

\autor{Nome do Autor}

\autorA{Nome do Autor A} % Em caso de TCC
\autorB{Nome do Autor B} % Em caso de TCC
\autorC{Nome do Autor C} % Em caso de TCC

\citaautores{AUTOR, Nome;} % Em caso de Tese ou Dissertação
\citaautores{AUTOR A, Nome; AUTOR B, Nome; AUTOR C, Nome;} % Em caso de TCC


\local{Curitiba}
\data{\the\year}
\orientador{Nome do orientador}
\coorientador{Nome do Coorientador}

\instituicao{Universidade Tecnológica Federal do Paraná}
\departamento{Departamento Acadêmico de Eletrotécnica}
\programa{Programa de Pós--Graduação em Engenharia Elétrica e Informática Industrial}


\instituicaoEng{Universidade Tecnológica Federal do Paraná}
\departamentoEng{Departamento Acadêmico de Eletrotécnica}
\programaEng{Programa de Pós--Graduação em Engenharia Elétrica e Informática Industrial}
\tipotrabalhoEng{Tipo do Trabalho} % TCC, Dissertação ou Tese


\siglainstituicao{UTFPR}
\sigladepartamento{DAELT}
\siglaprograma{CPGEI}


\tipotrabalho{Tipo do Trabalho} % TCC, Dissertação ou Tese
% O preambulo deve conter o tipo do trabalho, o objetivo, 
% o nome da instituição e a área de concentração 
\preambulo{Modelo canônico de trabalho monográfico acadêmico em conformidade com
as normas ABNT apresentado à comunidade de usuários \LaTeX.}
% ---

\preambulo{Proposta de Trabalho de Conclusão de Curso de Graduação, apresentado à disciplina de Metodologia Aplicada ao TCC, do curso de Engenharia de Controle e Automação do Departamento Acadêmico de Eletrotécnica (DAELT) da Universidade Tecnológica Federal do Paraná (UTFPR) como requisito para obtenção do título de Engenheiro de Controle e Automação.}


% ---
% Configurações de aparência do PDF final
% ---

%\definecolor{figcolor}{rgb}{1,0.4,0}  % orange
%\definecolor{tabcolor}{rgb}{1,0.4,0}  % orange
%\definecolor{eqncolor}{rgb}{1,0.4,0}  % orange
\definecolor{linkcolor}{rgb}{1,0.4,0}  % orange
%\definecolor{citecolor}{rgb}{1,0.4,0}  % orange
%\definecolor{seccolor}{rgb}{0,0,1}  % blue
%\definecolor{abscolor}{rgb}{0,0,1}  % blue
%\definecolor{titlecolor}{rgb}{0,0,1}  % blue
%\definecolor{biocolor}{rgb}{0,0,1}  % blue

% alterando o aspecto da cor azul
\definecolor{blue}{RGB}{41,5,195}

% informações do PDF
\makeatletter
\hypersetup{
     	%pagebackref=true,
		pdftitle={\@title}, 
		pdfauthor={\@author},
    	pdfsubject={\imprimirpreambulo},
	    pdfcreator={LaTeX with abnTeX2-utfpr},
		pdfkeywords={latex}{abntex}{abntex2}{\imprimirtipotrabalho}, 
		colorlinks=true,       		% false: boxed links; true: colored links
    	linkcolor=blue,          	% color of internal links
    	citecolor=blue,        		% color of links to bibliography
    	filecolor=magenta,      		% color of file links
		urlcolor=blue,
		bookmarksdepth=4
}
\makeatother
% --- 



% ---
% Estilo de capítulos
%
%\chapterstyle{default}
%\chapterstyle{pedersen} 
%\chapterstyle{lyhne} 
\chapterstyle{madsen} 
% \chapterstyle{veelo} 
%\chapterstyle{companion}

%\chapterstyle{thatcher}
%\chapterstyle{verville}
%\chapterstyle{VZ14} % Ver classe para maiores detalhes


%
% Veja outros estilos em:
% http://mirror.utexas.edu/ctan/info/latex-samples/MemoirChapStyles/MemoirChapStyles.pdf
% ---


% --- 
% Espaçamentos entre linhas e parágrafos 
% --- 

% O tamanho do parágrafo é dado por:
\setlength{\parindent}{1.3cm}

% Controle do espaçamento entre um parágrafo e outro:
\setlength{\parskip}{0.0cm}  % tente também \onelineskip

% ---
% compila o indice
% ---
\makeindex
% ---


%\includeonly{Proposta/proposta-de-tcc}
%\includeonly{Proposta/descricao-da-proposta}


% ----
% Início do documento
% ----
\begin{document}

% Seleciona o idioma do documento (conforme pacotes do babel)
%\selectlanguage{english}
\selectlanguage{brazil}

% Retira espaço extra obsoleto entre as frases.
\frenchspacing 

% ----------------------------------------------------------
% ELEMENTOS PRÉ-TEXTUAIS
% ----------------------------------------------------------
% \pretextual


% ---
% Capa
% ---
%\imprimircapaPOS
\imprimircapaGRAD

%\imprimircapaPOSnoBack % Capa sem imagem de fundo
%\imprimircapaGRADnoBack % Capa sem imagem de fundo


%\imprimircapautfpr
% ---

% ---
% Folha de rosto
% (o * indica que haverá a ficha bibliográfica)
% ---
\imprimirfolhaderosto*



% ----------------------------------------------------------
% ELEMENTOS TEXTUAIS
% ----------------------------------------------------------
\textual

% ----------------------------------------------------------
% Introdução (exemplo de capítulo sem numeração, mas presente no Sumário)
% ----------------------------------------------------------
% PROPOSTA DE TRABALHO DE CONCLUSÃO DE CURSO-----------------------------------------------------------

\chapter{PROPOSTA DE TRABALHO DE CONCLUSÃO DE CURSO}
\label{chap:proposta}

\section{TÍTULO}
\label{sec:titulo}
% Informe o título do trabalho-------------------------------------------------------------------------
O título é um texto, com poucas palavras, que deve expressar claramente: O objeto de investigação relativo ao tema e o que vai fazer (substitua este texto pelo título do trabalho).
%------------------------------------------------------------------------------------------------------

\section{MODALIDADE DO TRABALHO}
\label{sec:modalidade}
% Indique a Modalidade do Trabalho---------------------------------------------------------------------
% Opções:
% - Pesquisa
% - Desenvolvimento de Sistemas
Desenvolvimento de Sistemas
%------------------------------------------------------------------------------------------------------

\section{ÁREA DO TRABALHO}
\label{sec:area}
% Indique a Área do Trabalho---------------------------------------------------------------------------
Definir a área em que o trabalho está incluído (substitua este texto pela área do trabalho).
%------------------------------------------------------------------------------------------------------

\section{RESUMO}
\label{sec:resumo}
% Resumo do Trabalho-----------------------------------------------------------------------------------
% (máximo de 200 palavras)
Um resumo deve informar a essência do projeto de maneira resumida, mas completa. Os leitores devem ter uma ideia razoavelmente clara do projeto após ter lido o resumo. Basicamente deve-se colocar informações referentes a finalidade da pesquisa, procedimentos que serão utilizados, observações e dados a serem coletados, resultados esperados (substitua este texto pelo resumo do trabalho).
%------------------------------------------------------------------------------------------------------



% DESCRIÇÃO DA PROPOSTA--------------------------------------------------------------------

\chapter{DESCRIÇÃO DA PROPOSTA}
\label{chap:descricao}

\section{INTRODUÇÃO}
\label{sec:introducao}
% Introdução-------------------------------------------------------------------------------
% (máximo de 1 página)
Parte inicial do texto, na qual devem constar o tema e a delimitacão do assunto tratado, objetivos da pesquisa e outros elementos necessários para situar o tema do trabalho, tais como: justificativa, procedimentos metodológicos (classificação inicial), embasamento teórico (principais bases sintetizadas) e estrutura do trabalho, tratados de forma sucinta.

Sugere-se fortemente que a  introdução contenha os seguintes itens expressos em parágrafos:

P1. Contextualização do Projeto

P2. Definição do Problema

P3. Relevância do Problema

P4. Justificativa

P5. Desafios do Projeto

P6. Contribuição

(substitua este texto pela introdução do trabalho)
%-----------------------------------------------------------------------------------------

\section{OBJETIVOS}
\label{sec:objetivos}
% (máximo de 1/2 página)

\subsection{Objetivo Geral}
\label{subsec:objgeral}
% Objetivo Geral--------------------------------------------------------------------------
O objetivo geral é tratado em seu sentido mais amplo e constitui a ação que conduzirá ao tratamento da questão abordada no problema de pesquisa, fazendo menção ao objeto de uma forma mais direta. O objetivo geral deve:

Conter descrição do que vai fazer, de forma precisa e objetiva;

Ser diretamente ligado ao título;

Ser mais detalhado que o título;

Resolver o problema proposto;

Ser Claro, Conciso e Completo (CCC) e deve ser verificável ao final do trabalho.

(substitua este texto pelo objetivo geral do trabalho)
%-----------------------------------------------------------------------------------------

\subsection{Objetivos Específicos}
\label{subsec:objespc}
% Objetivos Específicos-------------------------------------------------------------------
Os objetivos específicos apresentam, de forma pormenorizada, detalhada, as ações que se prentede alcançar e estabelecem estreita relação com as particularidades relativas à temática trabalhada. Os objetivos específicos devem:

Ser Claros, Concisos e Completos (CCC) e devem ser verificáveis ao final do trabalho;

Fazem parte do detalhamento do objetivo geral ;

Devem ser iniciados com o verbo no infinitivo;

Podem ser considerados com subprodutos do objetivo geral.

(substitua este texto pelos objetivos específicos do trabalho)
%-----------------------------------------------------------------------------------------

\section{ESTADO DA ARTE}
\label{sec:estadoarte}
% Estado da Arte--------------------------------------------------------------------------
% (máximo de 2 páginas)
Uma vez formulado o problema a ser atacado, é preciso se inteirar do que já foi feito, dito e discutido sobre ele. Isso se chama "estado da arte". Pode ser que a dúvida, que está motivando a pesquisa, já tenha sido respondida de alguma maneira por alguém. Por isso, é preciso aprofundar o conhecimento sobre a questão, antes de dar prosseguimento ao projeto.

Essa etapa também recebe o nome de revisão bibliográfica, quando são estudados os trabalhos que se situam na circunvizinhança do problema, trabalhos que versam sobre problemas similares.

Vê-se aí por que a revisão bibliográfica é importante. De um lado, ela deve comprovar que o pesquisador não está querendo realizar algo que já foi feito, de outro lado, ela ajuda a encaminhar o passo seguinte da pesquisa, a justificativa, quer dizer, a argumentação sobre a relevância do trabalho.

Para a proposta de TCC deve ser descrito, de maneira breve, alguns (sugestão de 2 (dois) a 3 (três)) trabalhos correlatos, converse com seu orientador para citar os mais relevantes do tema abordado. Pode ser seguido a seguinte sugestão de parágrafos/tópicos:

P1. Descrição do trabalho 1

P2. Descrição do trabalho 2

P3. Descrição do trabalho 3

P4. Discussão dos trabalhos mencionados destacando porque eles são importantes para o trabalho proposto.

Para utilização de citações atente ao tipo de citação que se deseja usar. As citações são classificadas em indeireta e direta, podem ser longas ou curtas.

Uma citação indireta é a transcrição, com suas próprias palavras, das idéias de um autor, mantendo-se o sentido original. A citação indireta é a maneira que o pesquisador tem de ler, compreender e gerar conhecimento a partir do conhecimento de outros autores. Quanto à chamada da referência, ela pode ser feita de duas maneiras distintas, conforme o nome do(s) autor(es) façam parte do seu texto ou não. Exemplo de chamada fazendo parte do texto:\\
\\Enquanto \citeonline{Maturana2003} defendem uma epistemologia baseada na biologia. Para os autores, é necessário rever \ldots.\\

A chamada de referência foi feita com o comando \verb|\citeonline{chave}|, que produzirá a formatação correta.

A segunda forma de fazer uma chamada de referência deve ser utilizada quando se quer evitar uma interrupção na sequência do texto, o que poderia, eventualmente, prejudicar a leitura. Assim, a citação é feita e imediatamente após a obra referenciada deve ser colocada entre parênteses. Porém, neste caso específico, o nome do autor deve vir em caixa alta, seguido do ano da publicação. Exemplo de chamada não fazendo parte do texto:\\
\\Há defensores da epistemologia baseada na biologia que argumentam em favor da necessidade de \ldots \cite{Maturana2003}.\\

Nesse caso a chamada de referência deve ser feita com o comando \verb|\cite{chave}|, que produzirá a formatação correta.

Uma citação direta é a transcrição ou cópia de um parágrafo, de uma frase, de parte dela ou de uma expressão, usando exatamente as mesmas palavras adotadas pelo autor do trabalho consultado.

Quanto à chamada da referência, ela pode ser feita de qualquer das duas maneiras, assim como nas nas citações indiretas, conforme o nome do(s) autor(es) façam parte do texto ou não. Há duas maneiras distintas de se fazer uma citação direta, conforme o trecho citado seja longo ou curto.

Quando o trecho citado é longo (4 ou mais linhas) deve-se usar um parágrafo específico para a citação, na forma de um texto recuado (4 cm da margem esquerda), com tamanho de letra menor e espaçamento entrelinhas simples. Exemplo de citação longa:
\\\begin{citacao}
Desse modo, opera-se uma ruptura decisiva entre a reflexividade filosófica, isto é a possibilidade do sujeito de pensar e de refletir, e a objetividade científica. Encontramo-nos num ponto em que o conhecimento científico está sem consciência. Sem consciência moral, sem consciência reflexiva e também subjetiva. Cada vez mais o desenvolvimento extraordinário do conhecimento científico vai tornar menos praticável a própria possibilidade de reflexão do sujeito sobre a sua pesquisa \cite[p.~28]{Silva2000}.
\end{citacao}

Para fazer a citação longa deve-se utilizar os seguintes comandos:
\begin{lstlisting}[language=tex]
\begin{citacao}
<texto da citacao>
\end{citacao}
\end{lstlisting}

No exemplo acima, para a chamada da referência o comando \verb|\cite[p.~28]{Silva2000}| foi utilizado, visto que os nomes dos autores não são parte do trecho citado. É necessário também indicar o número da página da obra citada que contém o trecho citado.

Quando o trecho citado é curto (3 ou menos linhas) ele deve inserido diretamente no texto entre aspas. Exemplos de citação curta:\\
\\A epistemologia baseada na biologia parte do princípio de que "assumo que não posso fazer referência a entidades independentes de mim para construir meu explicar" \cite[p.~35]{Maturana2003}.\\
\\A epistemologia baseada na biologia de \citeonline[p.~35]{Maturana2003} parte do princípio de que "assumo que não posso fazer referência a entidades independentes de mim para construir meu explicar".\\

Outros exemplos de comandos para as chamadas de referências e o resultado produzido por estes são:\\
\\\citeonline{Maturana2003} \ \ \  \verb|\citeonline{Maturana2003}|\\
\citeonline{Barbosa2004} \ \ \   \verb|\citeonline{Barbosa2004}|\\
\cite[p.~28]{Silva2000} \ \ \  \verb|\cite[p.~28]{Silva2000}|\\
\citeonline[p.~33]{Silva2000} \ \ \   \verb|\citeonline[p.~33]{v}|\\
\cite[p.~35]{Maturana2003} \ \ \   \verb|\cite[p.~35]{Maturana2003}|\\
\citeonline[p.~35]{Maturana2003} \ \ \   \verb|\citeonline[p.~35]{Maturana2003}|\\
\cite{Barbosa2004,Maturana2003} \ \ \   \verb|\cite{Barbosa2004,Maturana2003}|\\

Em relação as referências, a bibliografia é feita no padrão \textsc{Bib}\TeX{}. As referências são colocadas em um arquivo separado. Neste template as referências são armazenadas no arquivo "base-referencias.bib".

Existem diversas categorias documentos e materiais componentes da bibliografia. A classe abn\TeX{} define as seguintes categorias (entradas):

\begin{lstlisting}[language=tex]
@book
@inbook
@article
@phdthesis
@mastersthesis
@monography
@techreport
@manual
@proceedings
@inproceedings
@journalpart
@booklet
@patent
@unpublished
@misc
\end{lstlisting}

Cada categoria (entrada) é formatada pelo pacote \citeonline{abnTeX22014d} de uma forma específica. Para maiores detalhes, refira-se a \citeonline{abnTeX22014d}, \citeonline{abnTeX22014b}, \citeonline{abnTeX22014c}.
%-----------------------------------------------------------------------------------------

\section{DIFERENCIAL TECNOLÓGICO}
\label{sec:diferencial}
% Diferencial Tecnológico-----------------------------------------------------------------
% (máximo de ½ página)
O diferencial teórico é uma complementação do tópico discussão da seção estado da arte, onde será evidenciado qual o diferencial do trabalho perante os demais correlatos já existentes. Deve-se destacar os seguintes itens:

Diferencial do trabalho proposto perante produtos concorrentes ou semelhantes;

Vantagens que os possíveis usuários terão ao usar o trabalho a ser desenvolvido;

Destacar inovação tecnológica, por exemplo, uso de novas tecnologias e vantagens;

(substitua este texto pelo diferencial tecnológico do trabalho)
%-----------------------------------------------------------------------------------------

\section{PROCEDIMENTOS METODOLÓGICOS/METODOLOGIA} % Escolher o nome mais adequado ao trabalho
\label{sec:metodologia}
% Procedimentos Metodológicos/Metodologia-------------------------------------------------
% (máximo de 2 páginas)
Na seção de procedimentos metodológicos ou metodologia (ver qual o nome mais adequado ao trabalho) deve ser descrito sucintamente o procedimentos metodológicos para a execução do projeto ressaltando como os objetivos serão alcançados. 

Em geral, a seção descreve os procedimentos usados para resolver o problema atacado. Pode ser estruturada em tópicos, onde cada tópico representa um subproduto do objetivo geral.

No caso de desenvolvimento de sistemas deve-se descrever a metodologia a ser utilizada, por exemplo Scrum, eXtreme Programming, RUP, etc. 

Também pode ser descritos técnicas de desenvolvimento de software como por exemplo TDD, BDD, SPA,  etc.

(substitua este texto pelo de procedimentos metodológicos/metodologia do trabalho)
%-----------------------------------------------------------------------------------------

\section{CONCLUSÃO/CONSIDERAÇÕES FINAIS} % Escolher o nome mais adequado ao trabalho
\label{sec:conclusao}
% Conclusão/Considerações Finais----------------------------------------------------------
% (máximo de ½ página)
Na seção de Conclusão ou Considerações Finais (ver qual o nome mais adequado ao trabalho) o acadêmico deve descrever:

Como espera alcançar os objetivos propostos;

Destacar as dificuldades encontradas e previstas;

Fazer o fechamento do trabalho destacando sua importância.

(substitua este texto pelo de estado da arte do trabalho)
%-----------------------------------------------------------------------------------------

\section{PLANEJAMENTO DO TRABALHO}
\label{sec:planejamento}
% Planejamento do Trabalho----------------------------------------------------------------
% Esta seção não precisa ser editada, apenas edite o quadro 1 armazenada no diretório ".\dados\quadros"
O planejamento do trabalho de estágio que será desenvolvido pelo aluno, ao longo do período letivo, está descrito no cronograma da \tabref{tab:tabela01}. Neste cronograma constam todas as atividades com seus respectivos prazos para o cumprimento.


\begin{table}[!htb]
	%\centering
	\caption{Cronograma de Atividades.\label{tab:tabela01}}
	\begin{tabular}{p{4.5cm}|p{0.7cm}|p{0.7cm}|p{0.7cm}|p{0.7cm}|p{0.7cm}|p{0.7cm}|p{0.7cm}|p{0.7cm}|p{0.7cm}|p{0.7cm}}
		\hline
		\textbf{Atividades} & \textbf{Mar} & \textbf{Abr} & \textbf{Mai} & \textbf{Jun} & \textbf{Jul} & \textbf{Ago} & \textbf{Set} & \textbf{Out} & \textbf{Nov} & \textbf{Dez} \\
		\hline
		\small{1. Revisão dos apontamentos da banca} &   &   &   &   &   &   &   &   &   &  \\
		\hline
		\small{2. Revisão bibliográfica} &   &   &   &   &   &   &   &   &   &  \\
		\hline
		\small{3. Redação do projeto de TCC} &   &   & X & X &   &   &   &   &   &  \\
		\hline
		\small{4. Defesa do projeto de TCC} &   &   &   &   & X &   &   &   &   &  \\
		\hline
		\small{5. Escrita da Monografia de TCC} &   &   &   &   &   & X & X  & X &   &  \\
		\hline
		\small{6. Elaboração da apresentação final} &   &   &   &   &   &   &   & X & X &  \\
		\hline
		\small{7. Defesa final do TCC} &   &   &   &   &   &   &   &   & X &  \\
		\hline
	\end{tabular}
\end{table}


%-----------------------------------------------------------------------------------------

\section{RECURSOS NECESSÁRIOS}
\label{sec:recursos}
% Recursos Necessários--------------------------------------------------------------------
Coloque todos os materiais que serão utilizados. Exemplos: computadores, equipamentos de redes, licenças de software, etc. Também deverá ser colocado se os recursos estarão disponíveis. A universidade não comprará os recursos, portanto a responsabilidade de comprar algo será do aluno. (substitua este texto pelo de recursos necessários do trabalho)
%-----------------------------------------------------------------------------------------

\section{HORÁRIO DE TRABALHO}
\label{sec:horário}
% Horário de Trabalho---------------------------------------------------------------------
% Esta seção não precisa ser editada, apenas edite o quadro 2 armazenada no diretório ".\dados\quadros"
O horário destinado para realização das atividades do TCC, bem como o horário destinado para a reunião semanal/quinzenal com o orientador estão descritos no cronograma da \tabref{tab:tabela02}. Este horário é definido com orientador levando em consideração a complexidade do trabalho a ser desenvolvido.


\begin{table}[!htb]
	\centering
	\caption{Horário de Trabalho.\label{tab:tabela02}}
	\begin{tabular}{c|c|c|c|c|c|c}
		\hline
		\textbf{Horário} & \textbf{Seg} & \textbf{Ter} & \textbf{Qua} & \textbf{Qui} & \textbf{Sex} & \textbf{Sab} \\
		\hline
		\small{07h30 - 08h20} &   &   &   &   &   &   \\
		\hline
		\small{08h20 - 09h10} &   &   &   &   &   &   \\
		\hline
		\small{09h10 - 10h00} &   &   &  &  &   &   \\
		\hline
		\small{10h10 - 11h00} &   &   &   &   &  &   \\
		\hline
		\small{11h00 - 11h50} &   &   &   &   &   & \\
		\hline
		\small{} &   &   &   &   &   & \\
		\hline
		\small{13h00 - 13h50} &   & TCC & Orientaçao & TCC &   & \\
		\hline
		\small{13h50 - 14h40} &   & TCC & TCC &   &   & \\
		\hline
		\small{14h40 - 15h30} &   & TCC & TCC & TCC &   & \\
		\hline
		\small{15h40 - 16h30} &   & TCC & TCC & TCC &   & \\
		\hline
		\small{16h30 - 17h20} &   &   &   &   &   & \\
		\hline
		\small{17h20 - 18h10} &   &   &   &   &   & \\
		\hline
		\small{} &   &   &   &   &   & \\
		\hline
		\small{18h50 - 19h40} &   &   &   &   &   & \\
		\hline
		\small{19h40 - 20h30} &   &   &   &   &   & \\
		\hline
		\small{20h30 - 21h20} &   &   &   &   &   & \\
		\hline
		\small{21h30 - 22h15} &   &   &   &   &   & \\
		\hline
	\end{tabular}
\end{table}




%
%
%\section{Cronograma de atividades}
%
%O cronograma de atividades para esta tese é apresentado na Figura \ref{fig:cronogramaatividades}.
%%
%\begin{sidewaysfigure}
%	\begin{ganttchart}[x unit=0.48cm,
%		y unit title=0.4cm,
%		y unit chart=0.5cm,
%		vgrid,
%		hgrid, 
%		%title label anchor/.style={below=-1.6ex},
%		%title left shift=.05,
%		%title right shift=-.05,
%		title height=1,
%		%progress=today,
%		today=18,
%		today label={Qualificação},
%		bar/.append style={fill=gray!80},
%		bar incomplete/.append style={fill=gray!20},
%		group incomplete/.append style={draw=black,fill=none},
%		progress label text={},
%		%bar/.style={fill=gray!50},
%		%incomplete/.style={fill=white},
%		bar height=0.6,
%		bar top shift=.2,
%		group right shift=0.15,
%		group left shift=-0.15,
%		group top shift=.3,
%		group height=.3,
%		%group peaks={.15}{.3}{.1}
%		]{36}
%		
%		
%		%labels
%		\gantttitle{2011}{3} 
%		\gantttitle{2012}{9} 
%		\gantttitle{2013}{9} 
%		\gantttitle{2014}{9} 
%		\gantttitle{2015}{6}\\
%		\gantttitle{$3^o$ Tri.}{3} 
%		\gantttitle{$1^o$ Tri.}{3} \gantttitle{$2^o$ Tri.}{3} \gantttitle{$3^o$ Tri.}{3} 
%		\gantttitle{$1^o$ Tri.}{3} \gantttitle{$2^o$ Tri.}{3} \gantttitle{$3^o$ Tri.}{3} 
%		\gantttitle{$1^o$ Tri.}{3} \gantttitle{$2^o$ Tri.}{3} \gantttitle{$3^o$ Tri.}{3} 
%		\gantttitle{$1^o$ Tri.}{3} \gantttitle{$2^o$ Tri.}{3}\\
%		
%		\ganttbar[progress=100]{A.}{1}{9} \\
%		\ganttbar[progress=15/18*100]{B.}{4}{21} \\
%		\ganttbar[progress=12/30*100]{C.}{7}{36} \\
%		\ganttbar[progress=100]{D.}{7}{15} \\
%		\ganttbar[progress=100]{E.}{10}{15} \\
%		\ganttbar[progress=12/15*100]{F.}{7}{21} \\
%		\ganttbar[progress=3/9*100]{G.}{16}{24} \\
%		\ganttbar[progress=0]{H.}{19}{24} \\
%		\ganttbar[progress=0]{I.}{19}{30} \\
%		\ganttbar[progress=0]{J.}{25}{36} \\
%		\ganttbar[progress=0]{K.}{22}{36} \\
%		
%		
%	\end{ganttchart}
%	\caption{Cronograma de atividades.}
%	\label{fig:cronogramaatividades}
%\end{sidewaysfigure}

%\begin{enumerate}[A.]%
%	\item Créditos em disciplinas;
%	\item Estágio de docência: EEL7200 Eletrônica de Potencia II;
%	\item Revisão bibliográfica;
%	\item Estudo inicial do tema proposto, ou seja, estudo da modulação mais adequada e estudo do equilíbrio de potência entre os conversores;
%	\item Especificação, projeto, montagem de um protótipo monofásico e obtenção de resultados experimentais;
%	\item Redação do texto para o exame de qualificação e defesa;
%	\item Redação de artigos para congressos e periódicos referentes à versão monofásica;
%	\item Estudo da versão trifásica;
%	\item Especificação, projeto, montagem de um protótipo trifásico e obtenção de resultados experimentais;
%	\item Redação de artigos para congressos e periódicos referentes à versão trifásica;
%	\item Redação da versão final da tese de doutorado e defesa pública.
%\end{enumerate}


%
%\begin{sidewaysfigure}
%	
%	\centering
%	\begin{ganttchart}[y unit title=0.4cm,
%		y unit chart=0.5cm,
%		vgrid,hgrid, 
%		title label anchor/.style={below=-1.6ex},
%		title left shift=.05,
%		title right shift=-.05,
%		title height=1,
%		bar/.style={fill=gray!50},
%		incomplete/.style={fill=white},
%		progress label text={},
%		bar height=0.7,
%		group right shift=0,
%		group top shift=.6,
%		group height=.3,
%		group peaks height =.2
%		]
%		{1}{34}
%		%labels
%		\gantttitle{Project}{34} \\
%		\gantttitle{Sep}{1} 
%		\gantttitle{Oct}{4} 
%		\gantttitle{Nov}{4} 
%		\gantttitle{Dec}{4} 
%		\gantttitle{Jan}{4} 
%		\gantttitle{Feb}{4} 
%		\gantttitle{Mar}{4} 
%		\gantttitle{Apr}{4} 
%		\gantttitle{May}{4}  
%		\gantttitle{Jun}{1}\\
%		%tasks
%		\ganttbar{Outline Proposal}{1}{2} \\
%		\ganttbar{Research}{3}{8} \\
%		\ganttbar{Literature Review}{9}{10} \\
%		\ganttbar{Ethics Test}{11}{15} \\
%		\ganttbar[progress=33]{Progress Report}{20}{22} \\
%		\ganttbar{Final Report}{18}{19} \\
%		\ganttbar{task 7}{16}{18} \\
%		\ganttbar[progress=0]{Demonstration}{21}{34}
%		
%		%relations
%		
%		\ganttlink{elem0}{elem1} 
%		\ganttlink{elem0}{elem3} 
%		\ganttlink{elem1}{elem2} 
%		\ganttlink{elem3}{elem4} 
%		\ganttlink{elem1}{elem5} 
%		\ganttlink{elem3}{elem5} 
%		\ganttlink{elem2}{elem6} 
%		\ganttlink{elem3}{elem6} 
%		\ganttlink{elem5}{elem7} 
%	\end{ganttchart}
%	
%	\caption{Gantt Chart - Workplan}
%	
%\end{sidewaysfigure}
%
%\begin{ganttchart}[
%	hgrid,
%	vgrid,
%	x unit=18mm,
%	time slot format=little-endian
%	]{7.1.2013}{11.1.2013}
%	\gantttitlecalendar{weekday=shortname}\\
%%	\ganttbar{A.}{2} \\
%\end{ganttchart}


%\begin{ganttchart}[
%	hgrid,
%	vgrid,
%	x unit=18mm,
%	time slot format=little-endian
%	]{7.1.2013}{13.1.2013}
%	\gantttitlecalendar*{7.1.2013}{13.1.2013}{weekday=shortname}\\
%%	\gantttitle{Hora}{4}
%\end{ganttchart}



%\begin{ganttchart}[
%	hgrid,
%	vgrid,
%	x unit=10mm,
%	time slot format=isodate-yearmonth,
%	compress calendar=true
%	]{2017-01}{2017-12}
%	\gantttitlecalendar{year,month=shortname}
%	
%\end{ganttchart}

\section{EXEMPLOS DE DIAGRAMAS DE GRANT}

The timelines for the proposed work is presented in the following charts.
\autoref{fig:AdmTimeline} shows a broad overview of the project milestones and accomplishements.
In \autoref{fig:CompletedWorkTimeline} the energy deposition and neutronics optimization by genetic algothrims are shown.
Finally \autoref{fig:LightTransportTimeline} presents the timeline for simulation of light transport.
%%%%%%% ADMINISTRATIVE TIMELINE %%%%%
\begin{sidewaysfigure}
	\begin{center}
		\begin{ganttchart}[x unit=8.5mm,vgrid,time slot format=isodate,compress calendar,today=2013-08-05,today offset=.5,today label=Current Week,today rule/.style={draw=blue,ultra thick}]{2012-06-12}{2013-12-30}
			\gantttitlecalendar{year,month=shortname} \\
			\ganttgroup[progress=100]{Energy Modeling}{2012-06-15}{2013-06-01}\\
			\ganttgroup[progress=90]{Neutronics Optimization}{2012-12-15}{2013-3-30}\\
			\ganttgroup[progress=75]{GEANT4 Light Transport}{2013-4-1}{2013-9-1}\\
			\ganttgroup[progress=50]{Write Up and Conclusions}{2013-7-1}{2013-10-1}\\
			\ganttmilestone{Dissertation Proposal}{2013-8-9}\\
			\ganttbar{Disserrtation Defense}{2013-10-1}{2013-11-1}\\
			\ganttmilestone{Dissertation to Trace}{2013-11-27}\\
			\ganttmilestone{Pass/Fail Form}{2013-11-27}\\
		\end{ganttchart}
	\end{center}
	\caption{Proposed Adminstrative Timeline}
	\label{fig:AdmTimeline}
\end{sidewaysfigure}
%%%%%%% COMPLETED WORK %%%%%%%
\begin{sidewaysfigure}
	\begin{center}
		\begin{ganttchart}[x unit=8.5mm,vgrid,time slot format=isodate,compress calendar,today=2013-08-05,today offset=.5,today label=Current Week,today rule/.style={draw=blue,ultra thick}]{2012-06-12}{2013-12-30}
			\gantttitlecalendar{year,month=shortname} \\
			\ganttgroup[progress=100]{Energy Modeling}{2012-06-15}{2013-06-01}\\
			\ganttbar[progress=100]{Energy Deposition}{2012-06-15}{2013-4-15} \\
			\ganttbar[progress=100]{Range}{2013-5-25}{2013-6-15} \\
			\ganttmilestone{IEEE Secondary Electron Paper}{2013-06-15}\\
			\ganttgroup[progress=90]{Neutronics Optimization}{2012-12-15}{2013-3-30}\\
			\ganttbar[progress=100]{MCNP Neutronics}{2012-12-15}{2013-1-15} \\
			\ganttbar[progress=100]{Scripted Input Decks}{2013-1-30}{2013-2-5}\\
			\ganttlinkedbar[progress=90]{Genetic Algorthims}{2013-2-15}{2013-3-10}\\
			\ganttmilestone{ANS Presentation}{2013-11-10}
		\end{ganttchart}
	\end{center}
	\caption[Timeline of Completed Work]{Timeline of Completed Work.}
	\label{fig:CompletedWorkTimeline}
\end{sidewaysfigure}
%%%% LIGHT TRANSPORT TIMELINE
\begin{sidewaysfigure}
	\begin{center}
		\begin{ganttchart}[x unit=8.5mm,vgrid,time slot format=isodate,compress calendar,today=2013-08-05,today offset=.5,today label=Current Week,today rule/.style={draw=blue,ultra thick}]{2012-06-12}{2013-12-30}
			\gantttitlecalendar{year,month=shortname} \\
			\ganttgroup[progress=75]{GEANT4 Validation}{2013-1-10}{2013-7-10}\\
			\ganttbar[progress=100]{Flir Detectors}{2013-4-15}{2013-4-30}\\
			\ganttbar[progress=100]{Fabrication of Detectors}{2013-1-10}{2013-7-10} \\
			\ganttlinkedbar[progress=100]{GS20 Simulation}{2013-06-1}{2013-7-10}\\
			\ganttlinkedbar[progress=80]{Layred Simulations}{2013-6-15}{2013-7-15}\\
			\ganttgroup[progress=75]{GEANT4 RPM Light Transport}{2013-6-1}{2013-10-1}\\
			\ganttbar[progress=50]{Prelimiary Simuation}{2013-7-1}{2013-7-20} \\
			\ganttlinkedbar[progress=25]{Parameter Studies}{2013-8-1}{2013-9-1}\\
			\ganttlinkedbar[progress=0]{Optimization}{2013-9-1}{2013-10-1}\\
		\end{ganttchart}
	\end{center}
	\caption{Light Transport Timeline}
	\label{fig:LightTransportTimeline}
\end{sidewaysfigure}

%-----------------------------------------------------------------------------------------





% ----------------------------------------------------------
% ELEMENTOS PÓS-TEXTUAIS
% ----------------------------------------------------------
\postextual
% ----------------------------------------------------------

% ----------------------------------------------------------
% Referências bibliográficas
% ----------------------------------------------------------
%\bibliography{Refs/abntex2-modelo-references}
\bibliography{Refs/base-referencias}

\end{document}
